\documentclass[lualatex,aspectratio=169,unicode, 12pt]{beamer}

\usepackage{luatexja}
\usepackage{here, amsmath, latexsym, amssymb, bm, ascmac, mathtools, multicol, tcolorbox, subfig}

\usetheme{Boadilla}
% \usetheme[shape=square]{LightTheme}

\title{タイトル}
\subtitle{サブタイトル}
\author[著者略称]{作者}
\institute[所属略称]{所属}
\date{\today}

\usepackage{amsmath}	% required for `\align' (yatex added)
\begin{document}

\frame{\maketitle}

\begin{frame}{スライド}
 \begin{equation}
  \frac{1}{s^{2}}\frac{\partial^{2} u}{\partial t^{2}} = \frac{\partial^{2} u}{\partial x^{2}} + \frac{\partial^{2} u}{\partial y^{2}} + \frac{\partial^{2} u}{\partial z^{2}}
 \end{equation}
\end{frame}

\begin{frame}[plain]{目次}
 \tableofcontents
\end{frame}

\section{目次の具体例}
\begin{frame}[plain]{スライド}
 arrayも使えます
 \begin{align}
  x &= a + 3 \\
  y &= b - 4
 \end{align}
\end{frame}

\section{箇条書き}
\begin{frame}[plain]{箇条書き}
 \begin{itemize}
  \item item 1
        \begin{enumerate}
         \item item1-1
         \item item1-2
        \end{enumerate}
  \item item 2
        \begin{enumerate}[I]
         \item item 2-1
         \item item 2-2
         \item item 2-3
        \end{enumerate}
 \end{itemize}
\end{frame}

\section{ブロック環境}
\begin{frame}[plain]{ブロック環境}
 \begin{block}{block}
  simple block
 \end{block}
 \begin{alertblock}{alertblock}
  alertblock
 \end{alertblock}
 \begin{exampleblock}{exampleblock}
  exampleblock
 \end{exampleblock}
\end{frame}

\begin{frame}[plain]{数学ブロック環境}
 \begin{theorem}[定義名]
  $e^{ix} = \cos x + i \sin x$
 \end{theorem}
 \begin{definition}[定義名]
  $e^{ix} \approx 1 + x$
 \end{definition}
 \begin{corollary}[系名]
  $a + b + c = 0$
 \end{corollary}
 \begin{proof}
  $\int_{-\infty}^{\infty} e^{-a x^{2}} = 2 \int_{0}^{\infty} e^{-a x^{2}}$
 \end{proof}
\end{frame}

\section{表}
\begin{frame}[plain]{表}
 \begin{table}
  \caption{Caption}
  \label{table:sample}
  \centering
  \begin{tabular}{ccc}
   \hline
   料理& 値段 & 場所 \\
   \hline \hline
   チキン & 200円 & 公園\\
   \hline
  \end{tabular}
 \end{table}
\end{frame}

\end{document}
